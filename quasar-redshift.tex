% to-do items:
% ------------
% - write text
% - deal with all items marked "TBD"

\documentclass[preprint]{aastex}
\usepackage{amssymb,amsmath,mathrsfs}
\newcounter{address}
\newcommand{\foreign}[1]{\textit{#1}}
\newcommand{\forbidden}[1]{{[}{\mathrm{#1}}{]}}
\newcommand{\OIII}{\forbidden{OIII}}
\newcommand{\mmatrix}[1]{\boldsymbol{#1}}
\newcommand{\inverse}[1]{{#1}^{-1}}
\newcommand{\transpose}[1]{{#1}^{\mathsf{T}}}
\newcommand{\covar}{\mmatrix{C}}
\newcommand{\evec}{\mmatrix{e}}
\newcommand{\fvec}{\mmatrix{f}}
\newcommand{\gvec}{\mmatrix{g}}
\newcommand{\invvar}{\inverse{\covar}}
\newcommand{\documentname}{\textsl{Note}}

\begin{document}
\title{Quasar redshift determination:\ A data-driven model approach}
\author{Paraskevi~Tsalmantza\altaffilmark{\ref{MPIA}},
        Michael~V.~Maseda\altaffilmark{\ref{MPIA},\ref{Caltech}},
        Joe~Hennawi\altaffilmark{\ref{MPIA},\ref{email}},
        David~W.~Hogg\altaffilmark{\ref{MPIA},\ref{CCPP}}}
\setcounter{address}{1}
\altaffiltext{\theaddress}{\stepcounter{address}\label{MPIA}
Max-Planck-Institut f\"ur Astronomie, K\"onigstuhl 17,
D-69117 Heidelberg, Germany}
\altaffiltext{\theaddress}{\stepcounter{address}\label{Caltech}
Caltech}
\altaffiltext{\theaddress}{\stepcounter{address}\label{email} To whom
correspondence should be addressed: \texttt{joe@mpia.de}}
\altaffiltext{\theaddress}{\stepcounter{address}\label{CCPP} Center
for Cosmology and Particle Physics, Department of Physics, New York
University, 4 Washington Place, New York, NY 10003}

\begin{abstract}
We present a method for measuring the redshifts of quasars by fitting
them with a data-driven model.  This is a a model of the space of
possible quasar spectra derived from a collection of quasars observed
by the Sloan Digital Sky Survey.  We show that the measurements
obtained with the model fits are better than those obtained by
cross-correlation with a fixed quasar template, and that they perform
well even when the narrow $\OIII$ lines are outside the spectroscopic
window.  TBD: results here.
\end{abstract}

\section{Introduction}

TBD: What is the problem and why is it important?

The standard data-driven models for quasars---or any other kind of
spectroscopic object in astronomy---is the highest-ranked principal
components from a principal components analysis (PCA) or equivalent.
The PCA has one advantage and a number of drawbacks.  The advantage is
that it is entirely data-driven: The construction of the PCA requires
no theoretical or external knowledge about the spectra being modeled;
it is a dimensionality reduction in the space of the observed spectra.

There are many drawbacks to PCA but the most important is that the PCA
returns the principal directions---the eigenvectors with maximum
eigenvalues---of the variance tensor of the data; this variance tensor
has contributions from intrinsic variation among spectra, and
contributions from observational noise.  That is, a direction in
spectrum space can enter into the top principal components because it
is a direction of great astrophysical variation, or because there is a
lot of noise in the observations along that direction, or both.  PCA
is agnostic about the \emph{source} of the variance, while astronomers
are not; astronomers want to know about the astrophysical processes
that generate the data \emph{prior} to the addition of observational
noise.

Other drawbacks to PCA include the following: It treats the data as
drawn from a linear subspace of the full spectral space.  This
assumption is unlikely to be true in any application.  It also has
trouble separating the spectral variation that comes from amplitude
changes (overall flux or luminosity changes) as distinct from
variations that come from shape changes in the spectra.  Various hacks
have been employed to deal with this, but many of them make the linear
subspace assumption even less valid than it was \foreign{a priori}.
Finally, PCA has no idea about prior information; it is just as happy
creating components with negative amplitudes as positive amplitudes
and the linear subspace therefore contains many quadrants, in general,
that represent spectra with completely unphysical properties (such as
negative emission lines and the like).

In this \documentname, we use a data-driven model for quasar spectra
that overcomes most---though not all---of the problems with PCA.

\section{Method}

The data---training or test set---will be a set of $N$ spectra $i$,
each of which is a vector $\fvec_i$ of $M$ fluxes $f_{ij}$ at $M$
wavelengths $\lambda_j$.  Associated with each flux $f_{ij}$ is an
inverse uncertainty variance $1/\sigma^2_{ij}$, which we will imagine
are the elements of a $M\times M$ diagonal inverse covariance matrix
$\invvar_{i}$.

The model is that any spectrum $\fvec_i$ can be written as a linear
sum of $K$ components $\gvec_k$ (each of which also has $M$ components
$g_{kj}$)
\begin{eqnarray}\displaystyle
\fvec_i &=& \sum_{k=1}^K a_{ik}\,\gvec_k + \evec_i \nonumber\\
f_{ij} &=& \sum_{k=1}^K a_{ik}\,g_{kj} + e_{ij}
\quad ,
\end{eqnarray}
where the $a_{ik}$ are coefficients, and the vector $\evec_i$ (with
$M$ components $e_{ij}$) is a noise vector, imagined to be a random
variable drawn from a $M$-dimensional Gaussian with zero mean and
variance tensor $\covar_i$.

We determine the components $\gvec_k$ by bi-linear least-square
fitting to a \emph{training set}.  The goal is to find the set of
coefficients $a_{ik}$ and spectral components $g_{kj}$ that jointly
minimize a total $\chi^2$ over all the $N$ training data:
\begin{eqnarray}\displaystyle
\chi^2 &=& \sum_{i=1}^N \sum_{j=1}^M \frac{1}{\sigma^2_{ij}}
 \,\left[f_{ij} - \sum_{k=1}^K a_{ik}\,g_{kj}\right]^2
\quad .
\end{eqnarray}

TBD: describe optimization methodology.

TBD: describe prior.

\section{Training and test data}

\section{Results}

\section{Discussion}

\acknowledgements
Thanks to people.  Thanks to SDSS.  Thanks to R.  Thanks to grants.

\end{document}
