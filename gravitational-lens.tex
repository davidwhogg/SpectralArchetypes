\documentclass[12pt]{article}
\newcommand{\facility}[1]{\textsl{#1}}
\newcommand{\foreign}[1]{\textsl{#1}}
 \newcommand{\apriori}{\foreign{a priori}}
\newcommand{\equationname}[1]{equation~(\ref{#1})}
\newcommand{\fluxvec}{\vec{f}}
\begin{document}

For each spectrum $i$, we generate $K+1$ mutually exclusive hypotheses:
The null hypothesis $S_i$ that spectrum $i$ only has significant flux
coming from a single redshift $z_i$ determined by the \facility{SDSS}
pipeline, and $K$ hypotheses $D_{ij}$ that spectrum $i$ has significant
flux coming from two redshifts, $z_i$ and another redshift
$z_j>z_i+\epsilon$.  In the context of this very restricted universe
of hypotheses, the odds ratio $\Omega_i$ for the null hypothesis is
\begin{equation}\label{eq:odds}
\Omega_i = \frac{\sum_{j=1}^K p(D_{ij}|\fluxvec_i,I)}{p(S_i|\fluxvec_i,I)}
 = \sum_{j=1}^K \left[\frac{p(\fluxvec_i|D_{ij},I)}{p(\fluxvec_i|S_i,I)}
 \,\frac{p(D_{ij}|I)}{p(S_i|I)}\right] = \sum_{j=1}^K\Omega_{ij}\quad,
\end{equation}
where the spectral flux data are represented by the vector
$\fluxvec_i$, the symbol $I$ represents all of the prior information
in the problem, including but not limited to the hypothesis
specification, the wavelengths and uncertainties associated with the
spectral flux data, and any other knowledge that the investigator
might have about the hypotheses prior to any data analysis.  We have
implicitly defined an individual-hypothesis odds ratio
\begin{equation}
\Omega_{ij} = \frac{p(\fluxvec_i|D_{ij},I)}{p(\fluxvec_i|S_i,I)}
  \,\frac{p(D_{ij}|I)}{p(S_i|I)}\quad.
\end{equation}
Because an individual spectrum is unlikely \apriori\ to show two
redshifts, the prior probabilities will have the asymmetry
\begin{equation}
\sum_{j=1}^K p(D_{ij}|I) \ll p(S_i|I) \quad,
\end{equation}
and it remains for us to decide how to set the relative prior
probabilities among the $K$ hypotheses $D_{ij}$.

We have split the sum in the odds $\Omega_i$ into a sum of individual
odds ratios $\Omega_{ij}$ because, as we will see, we need to estimate
the ratio for each $j$ individually; we can't just evaluate the total
numerator and denominator of $\Omega_i$ independently.  The reason for
this is the all-important \emph{spectral coverage}.  Each setting of
the pair $(z_i,z_j)$ limits differently the spectral range over which
the eigenspectra are both well determined.  Imagine that one of the
hypotheses $D_{ij}$ is only testable on some particular spectral range
$[\lambda_{\min},\lambda_{\max}]$.  The data in this spectral range
can be used to estimate the single-hypothesis odds ratio $\Omega_{ij}$
of hypothesis $D_{ij}$ to the null hypothesis $S_i$.  Clearly
hypotheses $D_{ij}$ that are testable with larger spectral ranges will
be better tested, but the fact that different hypotheses are subject
to tests of different strengths merely weakens---does not
invalidate---the total hypothesis test.

The SDSS spectra have well-understood and near-gaussian uncertainties.
Therefore, for each hypothesis $D_{ij}$ we can perform least-square
fitting ($\chi^2$ minimization) on the subset of $N_{ij}$ pixels in
the flux vector $\fluxvec_i$ that overlap the eigenspectra spectral
ranges for both redshifts $z_i$ and $z_j$.  If we perform the
least-square fit with $n$ eigenspectra at each redshift, then the odds
ratio can be approximated by a modified difference in $\chi^2$:
\begin{equation}
\ln\Omega_{ij}= \frac{1}{2}\,\left[\chi^2_i-\chi^2_{ij}-n\right]
 +\ln\frac{p(D_{ij}|I)}{p(S_i|I)} \quad,
\end{equation}
where we have taken the natural logarithm to simplify things,
$\chi^2_i$ is the minimum $\chi^2$ under hypothesis $S_i$,
$\chi^2_{ij}$ is the minimum $\chi^2$ under hypothesis $D_{ij}$, the
adjustment of $-n$ accounts for the fact that the $S_i$ fit has $n$
fewer parameters than the $D_{ij}$ fit, and the last term is the prior
ratio.  Importantly, in this odds-ratio expression, the $\chi^2$ fits
for hypotheses $S_i$ and $D_{ij}$ must have been performed \emph{over
the same $N_{ij}$ pixels in both cases}.  Even then, the expression is
something of an approximation, because it effectively assumes that the
data are affected by perfectly known, perfectly gaussian noise and
that one of the two hypotheses is capable of providing a good fit to
the data.

\end{document}
