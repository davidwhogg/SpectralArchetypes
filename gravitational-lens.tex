\documentclass[12pt]{article}
\newcommand{\facility}[1]{\textsl{#1}}
\newcommand{\equationname}[1]{equation~(\ref{#1})}
\newcommand{\fluxvec}{\vec{f}}
\begin{document}

For each spectrum $i$, we generate $K+1$ mutually exclusive hypotheses:
The null hypothesis $S_i$ that spectrum $i$ only has significant flux
coming from a single redshift $z_i$ determined by the \facility{SDSS}
pipeline, and $K$ hypotheses $D_{ij}$ that spectrum $i$ has significant
flux coming from two redshifts, $z_i$ and another redshift
$z_j>z_i+\epsilon$.  In the context of this very restricted universe
of hypotheses, the odds ratio $\Omega_i$ for the null hypothesis is
\begin{equation}\label{eq:odds}
\Omega_i = \frac{\sum_{j=1}^K p(D_{ij}|\fluxvec_i,I)}{p(S_i|\fluxvec_i,I)}
 = \sum_{j=1}^K \left[\frac{p(\fluxvec_i|D_{ij},I)}{p(\fluxvec_i|S_i,I)}
 \,\frac{p(D_{ij}|I)}{p(S_i|I)}\right] \quad,
\end{equation}
where the spectral flux data are represented by the vector
$\fluxvec_i$, and the symbol $I$ represents all of the prior
information in the problem, including but not limited to the
hypothesis specification, the wavelengths and uncertainties associated
with the spectral flux data, and any other knowledge that the
investigator might have about the hypotheses prior to any data
analysis.  On the right-hand side of \equationname{eq:odds}, we have
pulled the summation off of the numerator, because, as we will see, we
need to estimate the ratio for each $j$ individually; we can't just
evaluate the numerator and denominator independently.

The reason for this is the all-important \emph{spectral coverage}.
Each setting of the pair $(z_i,z_j)$ limits differently the spectral
range over which the eigenspectra are both well determined.

\end{document}
