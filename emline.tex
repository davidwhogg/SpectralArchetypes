\documentclass[12pt,letterpaper]{article}
\usepackage{amssymb,amsmath}
\newcommand{\aij}{a_{ij}}
\newcommand{\bij}{b_{ij}}
\newcommand{\chisqij}{\chi^2_{ij}}
\newcommand{\chisqmax}{\chi^2_{\mathrm{max}}}
\newcommand{\datavec}[1]{\vec{\boldsymbol{#1}}}
  \newcommand{\vlambda}{\datavec{\lambda}}
  \newcommand{\vfi}{\datavec{f}_{\!i}}
  \newcommand{\vfj}{\datavec{f}_{\!j}}
  \newcommand{\vfmodelij}{\datavec{g}_{ij}}
  \newcommand{\vfapproxij}{\datavec{h}_{ij}}
\newcommand{\datamatrix}[1]{\mathbb{#1}}
  \newcommand{\identity}{\datamatrix{I}}
  \newcommand{\reddening}{\datamatrix{A}}
  \newcommand{\invcovarj}{\datamatrix{C}^{-1}_{j}}
\newcommand{\transpose}{^{\!\textsf{T}}}

\begin{document}

We start with $N$ galaxy spectra indexed by $j$.  Each galaxy spectrum
$j$ provides an ordered set of $M$ line fluxes $f_{jk}$ for $M$ lines
at rest wavelengths $\lambda_k$.  In some of what follows, we will
represent the line fluxes and wavelengths with ``vectors'' $\vfj$ and
$\vlambda$.  We seek to replace the set of $N$ spectra with some
smaller subset such that every spectrum is either included in the
subset or else adequately \emph{represented} there by some included
spectrum.  Ideally we would like to find the minimum-sized subset
(though this problem is NP-hard in general).

What do we mean if we say that galaxy $i$ adequately represents galaxy
$j$?  For our purposes, we say that $i$ represents $j$ if a scaled and
dust-reddened (or dereddened) version of the fluxes $\vfi$ is a good
model (in a chi-squared sense) for the fluxes $\vfj$.  That is, we
create a model $\vfmodelij$ for fluxes $\vfj$ out of the fluxes $\vfj$
of the form
\begin{equation}
\vfmodelij= \aij\,\left[\reddening^{\bij}\right]\cdot\vfi
\quad,
\end{equation}
where $\aij$ an $\bij$ are free parameters, $\reddening$ is the
diagonal $M\times M$ matrix with a multiplicative reddening law
(details below) evaluated at the $M$ wavelengths $\vlambda$ on the
diagonal and exponentiated according to standard matrix rules (which
are trivial in the diagonal case), and we have treated the fluxes
$\vfi$ and $\vfmodelij$ as column vectors.  The parameter $\aij$ is an
overall flux scale, and (if the matrix $\reddening$ is properly
constructed) the parameter $\bij$ is a ratio of attenuation amplitudes
or $A_V$ values for the two spectra $A_{Vj}/A_{Vi}$.  The model
$\vfmodelij$ is considered a good model for $\vfj$ if
\begin{equation}
\chisqij\equiv \left[\vfj-\vfmodelij\right]\transpose
  \cdot\invcovarj
  \cdot\left[\vfj-\vfmodelij\right] < \chisqmax
\quad,
\end{equation}
where $\invcovarj$ is the inverse covariance matrix representing the
observational uncertainties for fluxes $\vfj$ and $\chisqmax$ is a
parameter of the method which will generally be set to something on
the order of $M$.  In what follows, we will say that $i$
\emph{represents} $j$ if we can find parameters $\aij$ and $\bij$ such
that $\chisqij < \chisqmax$.

This representation condition has some desirable properties.  The
first is that $i$ is only considered to represent $j$ if $i$
is---quantitatively---a good fit to $j$.  This permits interpretation
of the representation function in terms of probabilistic inference
(for example, we can set the parameter $\chisqmax$ on the basis of a
quantitative likelihood).  The second is that the representation
function is \emph{asymmetric}: If $i$ represents $j$, it is not
necessary that $j$ represent $i$.  The asymmetry enters at the point
at which the inverse covariance matrix $\invcovarj$ is used.  In
general, higher signal-to-noise spectra will represented by a smaller
fraction of other spectra, and lower signal-to-noise spectra will be
represented by a larger fraction of other spectra.  High
signal-to-noise spectra will therefore be favored in the construction
of the archetype or minimal representing set.  A third desirable
property for this problem is that reddening or attenuation by dust has
been removed (at least approximately), so we obtain representation
when the emission line properties are \emph{intrinsically} similar,
not just similar in appearance.  Finally, each spectrum represents
\emph{itself} (with possible exceptions given below); this is
reasonable, and ensures that we can generate a subset of spectra at
least one member of which is capable of representing each spectrum in
the original set of $N$.

All that said, we do perform some heuristic modifications to the most
simple or natural procedure to improve performance.  The first is that
we \emph{don't} perform the full non-linear least-squares fit for
every possible pairwise $\chisqij$.  This million-by-million set of
non-linear optimization problems would be exceedingly time-consuming.
We linearize the problem around $\bij=0$ to make the appoximate model
$\vfapproxij$:
\begin{equation}
\vfapproxij= \left[\aij\,\identity + \bij\,\ln\reddening\right]\cdot\vfi
\quad,
\end{equation}
where $\identity$ is the $M\times M$ identity matrix.  We then use the
parameters found by least-square fitting the linearized model in the
correct, non-linear computation of $\chisqij$.  This method is
approximate, but tends to penalize large values of $\bij$ (that is,
large differences in reddening or large ratios of $A_V$), where we don't
trust any reddening or attenuation law anyway.

The second heuristic modification is that we add to the measurement
uncertainties on fluxes $\vfj$ in quadrature an artificial error set
to a fraction $\epsilon$ of the fluxes $\vfj$ themselves.  We use this
to set an ``error floor'' below which we do not expect representation
to be precise.  This permits representation of high signal-to-noise
spectra by other high signal-to-noise spectra, and implicitly
represents our concern that there may be systematic errors or precise
spectrum-to-spectrum variations that are not either represented by the
error model or interesting physically.  With this addition, the method
has two free parameters, $\chisqmax$ and $\epsilon$, both related to
the precision we require for representation.

[what do we use for the reddening law?]

\end{document}
