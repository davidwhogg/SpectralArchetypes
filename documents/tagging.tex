\documentclass[12pt]{article}

\newcommand{\project}[1]{\textsl{#1}}

\begin{document}

\section*{Data-driven stellar age estimates and chemical tags}

\paragraph{introduction:}
The idea of chemical tagging is that every stellar birth environment
(every molecular cloud) are expected to have slightly different
chemical abundances in detail, such that stars with common origins but
different present-day phase-space locations can be identified by their
surface chemical abundances.  This potentially revolutionary
scientific program has not yet delivered results, in part because it
seems to require tens or hundreds of thousands of high resolution,
high signal-to-noise stellar spectra, and in part because it requires
stellar models good enough to make consistent measurements of
abundances across a wide range of spectral types (temperatures and
surface gravities).

There are a range of near-future projects obtaining the required
spectra, including \project{APOGEE} (cite) and \project{HERMES}
(cite).  There are efforts to analyze and improve stellar models
(cite).  However, it is possible at the present day---with spectra in
hand and without trustworthy models at all---to assess the feasiblity
of chemical tagging projects.  The idea of the present project is the
following: If sufficient tagging information resides within high
resolution and high signal-to-noise spectra, then that information
will be visible in the spectra themselves.  Models might be required
to \emph{interpret} the information, but they are not required to
\emph{identify its existence}.  Since abundance measurements are made
in the spectra, if the information to perform precise chemical tagging
is not present in the spectra, it cannot be present in any abundance
measurements derived therefrom.

In what follows, we use machine-learning methods that have been
designed to find structure in large, high dimensional data sets to
look for structure in stellar spectra.  We perform a set of staged
experiments, starting with simple projects to produce data-driven age
indicators, alpha-abundance indicators, and cluster-membership
indicators.  We finish with an assessment of the feasibility of the
most ambitious chemical tagging programs.

\paragraph{generalities:}
Classification description of the problem.

Regression description of the problem.

Density estimation description of the problem.

\paragraph{data and objectives:}

\paragraph{results and discussion:}

\end{document}
