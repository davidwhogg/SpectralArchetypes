\documentclass[12pt, preprint]{aastex}

\newcommand{\sectionname}{Section}
\newcommand{\project}[1]{\textsl{#1}}
\newcommand{\var}{\sigma^{2}}
\newcommand{\invvar}{\sigma^{-2}}

\begin{document}

\title{Data-driven stellar age estimates}
\author{
  David~W.~Hogg,
  Hans-Walter~Rix, others
}

\begin{abstract}
We perform simple emission-line-based (activity-based) age
determinations without the use of \emph{any} stellar models or model
spectra from libraries.  The method is based on K-nearest-neighbor
methods from machine learning.
\end{abstract}

\section{introduction}

...Age is indicated by emission lines in main-sequence dwarfs...

...In what follows, we use novel machine-learning-like methods that
have been designed to find structure in large, high dimensional data
sets to perform flexible fitting tasks with stellar spectra.

...Related prior work includes PCA, HMF, XD, SVM.  These all build
data-driven models of non-trivial data sets, and all could be used
here.  We are going to try to stick with methods that involve building
probabilistic generative models...

\section{method}

copied text: ...Regression description of the problem.  We have stars
in a range of ages, temperatures, surface gravities, reddenings,
distances, metallicities, and alpha abundances.  We either start with
known ages (the supervised version of the problem) or else we start
with an emission-line profile that indicates age (the unsupervised
version of the problem).  In the supervised version, we fit every star
as a base spectrum plus age indicator components.  In the unsupervised
version, we fit every star as a base spectrum plus known profile.  The
base spectrum, in each case, is a data-driven model, or the best you
can do with all the data you have ever seen.  HMF is good for this.
But if you want to marginalize out stellar base spectrum, then you
need to build priors.  Hierarchical is the best way to do that.

There are $N$ stars $n$, for each of which we have a spectrum $F_n$
(with dimensions of energy per time per area per wavelength).  Each
spectrum $F_n$ is represented as $L$ values $F_{n\ell}$ on a standard set
of rest-frame wavelengths $\lambda_\ell$.  These wavelengths are truly
rest-frame, because the \project{SDSS} spectral extraction pipelines
permit extraction of the spectrum shifted to the rest frame by the
(presumably well-measured) redshift.

Every measurement $F_{n\ell}$ comes with an uncertainty inverse
variance $\invvar_{n\ell}$.  We manipulate inverse variance rather
than ``sigma'' or variance because the inverse variance is the weight
for weighted least squares, and because missing data get zeros (not
infinities) in this well-behaved quantity.  In general, because of
observing and data analysis realities, all the weights
$\invvar_{n\ell}$ are different and not directly computable from the
$F_{n\ell}$ or any trivial combination of meta-data.  The data are
considered to be maximally heteroskedastic and arbitrarily censored,
but all in a known way.

In summary, the algorithm is:
\begin{enumerate}
\item For every pair $(m, n)$ of spectra, under the assumption that
  the two stars are identical in every respect except for distance and
  age, obtain the chi-squared minimum distance ratio $D_m/D_n$ by
  weighted linear least-square fitting.  This fit has one free
  parameter per pair.  The weight for each pixel $\ell$ in the fit is
  $[\var_{m\ell} + \var_{n\ell}]^{-1}$, but zeroed out in the spectral
  pixels in the neighborhood of the age-indicator emission lines.  In
  addition to the distance ratio, save $\chi^2_{mn}$, the chi-squared
  value at that minimum.  In principle this step can be sped up by
  only computing the distance-ratio fit for pairs of stars that have
  similar colors or derived properties in advance.  The width $d$ of
  the excluded region near the age-indictor lines is a free parameter
  of the method.
\item For each spectrum $n$, find the $K$ nearest neighbors $k$ in a
  chi-squared sense, using the minimum chi-squared values
  $\chi^2_{mk}$ as the ``distance''.  For our purposes here and below,
  the $K$ nearest neighbors \emph{includes} spectrum $n$.  $K$ is a
  free parameter of the method.
\item Each of the $K$ neighbor spectra $F_k$, plus the previously
  computed distance ratio $D_k/D_n$, effectively makes a prediction
  for the small number $d$ of spectral pixels $F_{n\ell}$ near the
  age-indicator lines.  We perform a weighted linear least-squares fit
  (weighting each pixel $k\ell$ by $\invvar_{k\ell}$), to the $d\,K$
  relevant spectral pixels with $d+K$ parameters: There are $d$
  spectral pixel values in the small wavelength region to represent
  the assumed common distance-corrected stellar spectral shape.  There
  are $K$ amplitudes for the age-indicator emission template spectrum,
  one per star.  Save all the fit parameters, $d+K$ for each of the
  $N$ spectra.
\item The age indicator for star $n$ is the age-indicator amplitude
  from the fit corresponding to the spectrum which was used as the
  central spectrum in the nearest-neighbor draw.  This means that the
  whole procedure needs to run $N$ times.  Things could be sped up
  with cleverness, but computers are cheap!
\end{enumerate}

\section{data and results}

\section{discussion}

\acknowledgements It is a pleasure to thank
  Ross Fadely,
  Rob Fergus,
  Dan Foreman-Mackey, and
  Dilip Krishnan
for valuable discussions.

\end{document}
