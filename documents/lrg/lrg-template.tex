% to-do items:
% ------------
% - write text
% - deal with all items marked "TBD"

% style notes:
% ------------
% - all equations in eqnarray \displaystyle
% - all citations in \citealt

\documentclass[preprint]{aastex}
\usepackage{amssymb,amsmath,mathrsfs}
\newcounter{address}
\newcommand{\foreign}[1]{\textit{#1}}
\newcommand{\project}[1]{\textsl{#1}}
\newcommand{\unit}[1]{\mathrm{#1}}
\newcommand{\km}{\unit{km}}
\newcommand{\s}{\unit{s}}
\newcommand{\kmps}{\km\,\s^{-1}}
\newcommand{\mum}{\unit{\mu m}}
\newcommand{\mmatrix}[1]{\boldsymbol{#1}}
\newcommand{\inverse}[1]{{#1}^{-1}}
\newcommand{\transpose}[1]{{#1}^{\mathsf{T}}}
\newcommand{\covar}{\mmatrix{C}}
\newcommand{\avec}{\mmatrix{a}}
\newcommand{\evec}{\mmatrix{e}}
\newcommand{\fvec}{\mmatrix{f}}
\newcommand{\Fvec}{\mmatrix{F}}
\newcommand{\gvec}{\mmatrix{g}}
\newcommand{\invvar}{\inverse{\covar}}
\newcommand{\all}[1]{\{{#1}\}}
\newcommand{\equationname}{equation}
\newcommand{\documentname}{\textsl{Note}}
\newcommand{\obs}{\mathrm{obs}}
\newcommand{\dd}{\mathrm{d}}
\newcommand{\like}{\mathscr{L}}

\begin{document}

%% \title{Data-driven templates for redshift determination}
%% \author{some permutation of PT\altaffilmark{\ref{MPIA}},
%%         DWH\altaffilmark{\ref{MPIA},\ref{CCPP}}}
%% \setcounter{address}{1}
%% \altaffiltext{\theaddress}{\stepcounter{address}\label{MPIA}
%% Max-Planck-Institut f\"ur Astronomie, K\"onigstuhl 17,
%% D-69117 Heidelberg, Germany}
%% \altaffiltext{\theaddress}{\stepcounter{address}\label{CCPP} Center
%% for Cosmology and Particle Physics, Department of Physics, New York
%% University, 4 Washington Place, New York, NY 10003}

%% \begin{abstract}
%%   We implement non-negative matrix factorization, adapted to
%%   heteroscedastic data, to generate template spectra for redshift
%%   determination.  We use the generated templates to determine
%%   redshifts for \project{SDSS-III} \project{BOSS} spectra.
%% \end{abstract}

%% \section{Introduction}

%% In this \documentname, we implement and execute a data-driven model
%% for luminous red galaxy spectra that overcomes most of the problems
%% associated with PCA-based spectral decompositions.

%% \section{Method}

The training data set will be a set of $N$ spectra $i$, each of which
is a vector $\fvec_i$ of $M$ fluxes $f_{ij}$ at $M$ wavelengths
$\lambda_j$.  Associated with each flux $f_{ij}$ is an inverse
uncertainty variance $1/\sigma^2_{ij}$, which we will imagine are the
elements of a $M\times M$ diagonal inverse covariance matrix
$\invvar_{i}$.

The model is that any spectrum $\fvec_i$ can be written as a linear
sum of $K$ components $\gvec_k$ (each of which also has $M$ components
$g_{kj}$)
\begin{eqnarray}\displaystyle
\fvec_i &=& \sum_{k=1}^K a_{ik}\,\gvec_k + \evec_i \nonumber\\
f_{ij} &=& \sum_{k=1}^K a_{ik}\,g_{kj} + e_{ij}
\quad,
\end{eqnarray}
where the $a_{ik}$ are coefficients, and the vector $\evec_i$ (with
$M$ components $e_{ij}$) is a noise vector, imagined to be a random
variable drawn from a $M$-dimensional Gaussian with zero mean and
(assumed known) variance tensor $\covar_i$.  To this model we add the
very powerful constraint of non-negativity in all aspects, to wit,
\begin{eqnarray}\displaystyle
a_{ik} &>& 0
\nonumber\\
g_{kj} &>& 0
\quad,
\end{eqnarray}
for all values of the indices $i$, $j$, and $k$.

In this model, the maximum-likelihood coefficients $a_{ik}$ and
spectral components $g_{kj}$ are those that jointly minimize a total
badness-of-fit scalar $\chi^2$ over all the $[M\,N]$ training data
values:
\begin{eqnarray}\displaystyle
\chi^2 &=& \sum_{i=1}^N 
 \,\transpose{\left[\fvec_i - \sum_{k=1}^K a_{ik}\,\gvec_k\right]}
 \cdot\invvar_i\cdot\left[\fvec_i - \sum_{k=1}^K a_{ik}\,\gvec_k\right]
 \nonumber\\
\chi^2 &=& \sum_{i=1}^N \sum_{j=1}^M \frac{1}{\sigma_{ij}^2}
 \,\left[f_{ij} - \sum_{k=1}^K a_{ik}\,g_{kj}\right]^2
\quad ,
\end{eqnarray}
where we have implicitly assumed that the covariance matrices are all
diagonal, with diagonal components $\sigma_{ij}^2$.

This problem can be seen as a form of matrix factorization, in which
we are trying to approximate the large non-square data matrix $f_{ij}$
as a product of two smaller matrices $a_{ik}$ and $g_{kj}$, subject to
the non-negativity constraint.  The best approximation is the one that
minimizes the badness-of-fit scalar $\chi^2$.  Non-negative matrix
factorization is a solved problem (for an astronomical example, see
\citealt{kcorrect}); briefly, optimization proceeds by starting
at an all-positive first guess and then iterating these multiplicative
updates:
\begin{eqnarray}\displaystyle
a_{ik} &\leftarrow& a_{ik}
 \,\left[\sum_{j=1}^M\frac{1}{\sigma_{ij}^2}\,f_{ij}\,g_{kj}\right]
 \,\left[\sum_{m=1}^K\sum_{j=1}^M\frac{1}{\sigma_{ij}^2}\,a_{im}\,g_{mj}\,g_{kj}\right]^{-1}
\nonumber\\
g_{kj} &\leftarrow& g_{kj}
 \,\left[\sum_{i=1}^N\frac{1}{\sigma_{ij}^2}\,f_{ij}\,a_{ik}\right]
 \,\left[\sum_{m=1}^K\sum_{i=1}^N\frac{1}{\sigma_{ij}^2}\,a_{ik}\,a_{im}\,g_{mj}\right]^{-1}
\quad.\label{eq:update}
\end{eqnarray}

The non-negative updates of \equationname~(\ref{eq:update}) are
guaranteed to maintain non-negativity if the inputs $f_{ij}$,
$a_{ik}$, and $g_{kj}$ are all non-negative initially.  There are two
possible catches: The first is that if a coefficient $a_{ik}$ or
spectral pixel $g_{kj}$ ever goes to zero or starts at zero, the
purely multiplicative updates can never ``resurrect'' it.  The second
is that the data often have---through noise or systematic error (for
spectra, think ``sky subtraction'')---negative values ($f_{ij}<0$ for
some $i,j$).  These must be addressed in a responsible way.  TBD: What
do we do and why?

Ordinarily, for any $K$-dimensional linear subspace, there are many
choices for the components $\gvec_k$: The components can be reordered,
multiplied by scalars, or replaced with linear combinations of
themselves.  Many of these degeneracies get broken by the non-negative
constraint, but not all.  We don't try to break all of the remaining
degeneracies, but we do enforce the constraint
\begin{eqnarray}\displaystyle
\gvec_k\cdot\gvec_k &=& M
\quad,
\end{eqnarray}
where the inner product is taken with the trivial (identity) metric in
the spectrum space, and we re-order the components (permute the $k$
values) to maintain
\begin{eqnarray}\displaystyle
\sum_{i=1}^N a_{ik} &>& \sum_{i=1}^N a_{i[k+1]}
\quad.
\end{eqnarray}

\begin{thebibliography}{70}
\bibitem[Blanton \& Roweis(2007)]{kcorrect}
Blanton,~M.~R., \& Roweis,~S., 2007, \aj, 133, 734 
\end{thebibliography}

\end{document}
