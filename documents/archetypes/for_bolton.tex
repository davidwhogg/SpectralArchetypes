\documentclass[12pt,letterpaper]{article}

\newcommand{\inverse}[1]{{#1}^{-1}}
\newcommand{\given}{\,|\,}
\newcommand{\setof}[1]{\left\{{#1}\right\}}
\newcommand{\normal}{N}

\begin{document}

Imagine that our model of quasar spectra is that every quasar $n$ is
generated by one of $K$ archetypes,
where the ``shape'' of each archetype $k$ is $x_k$,
and the amplitude (mean apparent brightness) of each quasar is $a_n$.
The first postulate of our model then is that our data $y_n$ on spectrum $n$ is given by
\begin{eqnarray}
y_n
  &=& a_n\,x_{k_n} + e_n
\quad ,
\end{eqnarray}
where $k_n$ is the archetype $k$ relevant to (or that generates) spectrum $n$,
and $e_n$ is a noise draw.
The next postulate is that the noise $e_n$ is drawn from a Gaussian
with known inverse covariance $\inverse{C_n}$.
This inverse covariance
probably contains many zeros (where data are missing),
and is probably (close to) diagonal,
though this latter assumption could be relaxed.
In this model, the likelihood function is
\begin{eqnarray}
p(y_n\given k_n, a_n, \setof{x_k}, C_n)
  &=& \normal(y_n\given a_n\,x_{k_n}, C_n)
\quad,
\end{eqnarray}
where $k_n$ is an integer parameter indicating the archetype $k$ appropriate to quasar $n$,
$a_n$ is the amplitude of quasar $n$,
$\setof{x_k}$ is the set of normalized (unit-amplitude) archetypes $x_k$,
and $\normal(x\given m,V)$ is the $D$-dimensional normal or Gaussian distribution
with $D$-vector mean $m$ and $D\times D$ variance matrix $V$.

We eschew righteousness and simply seek the maximum-likelihood values for the parameters
$a_n$, $k_n$, $\setof{x_k}$.
This problem is non-convex, so we further debase ourselves by looking only for a good
local optimum of the likelihood.
A procedure that is guaranteed to find an optimum is the follows:
\begin{enumerate}
\item At fixed $\setof{x_k}$, find by one-dimensional weighted least squares
the best-fit amplitude $a_{nk}$
for every spectrum $n$ given every possible archetype $x_k$.
Compute also for each spectrum--archetype pair the best-fit likelihood value.
\item For each spectrum $n$, choose the maximum-likelihood $k_n$,
given the best-fit likelihood values from the previous step.
\item At fixed $a_n$ and $k_n$, find by weighted least squares
the best-fit set of spectral archetypes $\setof{x_k}$.
Re-normalize those spectral archetypes to ``unit flux'' normalization.
\item Iterate to convergence.
\end{enumerate}
This procedure is almost certainly not the best procedure for optimizing the likelihood,
but it has the property that it increases the total likelihood at every step;
it is thus guaranteed to converge (except perhaps in pathological situations).

This is all very sensible, but there are many decisions to be made, not limited to the following:
\begin{itemize}
\item How do we set $K$?
Ideally cross-validation, but we could start by ``just sayin''.
\item What basis to work in?
We don't have to represent the $x_k$ in any particular basis;
it could be a spectral-pixel basis, or it could be some eigenspectral basis.
The default choice would be independent spectral pixels.
\item How do we initialize the $\setof{x_k}$?
Probably we should split the quasars up by color or something similar
(though definitely not redshift) and perform the $\setof{x_k}$ optimization first.
\item How do we build the $\inverse{C_n}$?
We use what is known about the noise in the system and missing data,
and possibly also apodize near the missing data.
\item Do we regularize any of the fits?
We probably should, just so the system doesn't go haywire where there are no data.
This regularization should be some kind of ``smoothness'' regularization.
\item How do we normalize the $x_k$?
If we don't normalize, the system has a possibly disastrous degeneracy.
The simplest choice is to choose a rest-frame bandpass and set
the mean archetype flux to unity within the bandpass.
\end{itemize}

\end{document}
