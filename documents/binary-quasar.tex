% to-do items:
% ------------
% - write text
% - deal with all items marked "TBD"

\documentclass[preprint]{aastex}
\usepackage{amssymb,amsmath,mathrsfs}
\newcounter{address}
\newcommand{\foreign}[1]{\textit{#1}}
\newcommand{\project}[1]{\textsl{#1}}
\newcommand{\SDSS}{\project{SDSS}}
\newcommand{\unit}[1]{\mathrm{#1}}
\newcommand{\km}{\unit{km}}
\newcommand{\s}{\unit{s}}
\newcommand{\kmps}{\km\,\s^{-1}}
\newcommand{\mum}{\unit{\mu m}}
\newcommand{\forbidden}[1]{{[}{\mathrm{#1}}{]}}
\newcommand{\OIII}{\forbidden{OIII}}
\newcommand{\mmatrix}[1]{\boldsymbol{#1}}
\newcommand{\inverse}[1]{{#1}^{-1}}
\newcommand{\transpose}[1]{{#1}^{\mathsf{T}}}
\newcommand{\covar}{\mmatrix{C}}
\newcommand{\avec}{\mmatrix{a}}
\newcommand{\bvec}{\mmatrix{b}}
\newcommand{\evec}{\mmatrix{e}}
\newcommand{\fvec}{\mmatrix{f}}
\newcommand{\gvec}{\mmatrix{g}}
\newcommand{\invvar}{\inverse{\covar}}
\newcommand{\all}[1]{\{{#1}\}}
\newcommand{\documentname}{\textsl{Note}}
\newcommand{\obs}{\mathrm{obs}}
\newcommand{\dd}{\mathrm{d}}
\newcommand{\like}{\mathscr{L}}

\begin{document}
\title{A catalog of binary quasars}
\author{some permutation of PT\altaffilmark{\ref{MPIA}},
        RD\altaffilmark{\ref{MPIA}},
        DWH\altaffilmark{\ref{MPIA},\ref{CCPP}}}
\setcounter{address}{1}
\altaffiltext{\theaddress}{\stepcounter{address}\label{MPIA}
Max-Planck-Institut f\"ur Astronomie, K\"onigstuhl 17,
D-69117 Heidelberg, Germany}
\altaffiltext{\theaddress}{\stepcounter{address}\label{CCPP} Center
for Cosmology and Particle Physics, Department of Physics, New York
University, 4 Washington Place, New York, NY 10003}

\begin{abstract}
  With a data-driven model of the space of possible quasar
  spectra---derived from the collection of all quasars observed by the
  \project{Sloan Digital Sky Survey} (\SDSS)---we fit every
  low-redshift ($z<$TBD) \SDSS\ quasar with a pair of quasars at
  substantial velocity separation.  We find XX new plausible binary
  quasar candidates and a raft of double-peaked broad-line quasars.
  The binary quasar candidates have the following physical properties:
  TBD.
\end{abstract}

\section{Introduction}

TBD: What is the problem and why is it important?  God tells us that
there will be many binary quasars; they have been hard to find.

TBD: Here's what's known: Four examples, all with issues.  Also, all
have the disturbing property that the broad-line component is shifted
to the red of the narrow-line component.

In this \documentname, we use a data-driven model for quasar spectra
constructed to explain the observed variations of quasars in the
\SDSS.  This model is more sophisticated than a principal components
analysis in that it takes proper account of the observational
uncertainties but is otherwise similar in spirit.

\section{Method}

We ``learn''---or ``find optimal''---spectral components or basis
spectra by a bilinear optimization on a training set of spectra.  We
use linear combinations of these components at two different
redshifts---the \SDSS\ redshift and a second redshift---to perform a
two-redshift against one-redshift hypothesis test.

\subsection{Training}

The training data set will be a set of $N$ \SDSS\ spectra $i$, each of
which is a vector $\fvec_i$ of $M$ fluxes $f_{ij}$ at $M$ wavelengths
$\lambda_j$ (logarithmically spaced; see below).  Associated with each
flux $f_{ij}$ is an inverse uncertainty variance $1/\sigma^2_{ij}$,
which we will imagine are the elements of a $M\times M$ diagonal
inverse covariance matrix $\invvar_{i}$.

The model is that any spectrum $\fvec_i$ can be written as a linear
sum of $K$ components $\gvec_k$ (each of which also has $M$ components
$g_{kj}$)
\begin{eqnarray}\displaystyle
\fvec_i &=& \sum_{k=1}^K a_{ik}\,\gvec_k + \evec_i \nonumber\\
f_{ij} &=& \sum_{k=1}^K a_{ik}\,g_{kj} + e_{ij}
\quad ,
\end{eqnarray}
where the $a_{ik}$ are coefficients, and the vector $\evec_i$ (with
$M$ components $e_{ij}$) is a noise vector, imagined to be a random
variable drawn from a $M$-dimensional Gaussian with zero mean and
variance tensor $\covar_i$.

We determine the components $\gvec_k$ by bi-linear least-square
fitting to a \emph{training set}.  The goal is to find the set of
coefficients $a_{ik}$ and spectral components $g_{kj}$ that jointly
minimize a total goodness-of-fit scalar $\chi^2$ over all the $N$
training data:
\begin{eqnarray}\displaystyle
\chi^2 &=& \sum_{i=1}^N \sum_{j=1}^M \frac{1}{\sigma^2_{ij}}
 \,\left[f_{ij} - \sum_{k=1}^K a_{ik}\,g_{kj}\right]^2
\quad .
\end{eqnarray}
This problem is not convex, and there are multiple minima, including
degenerate solutions.  We optimize to a local minimum using iterated
least squares.  In each step, we fix the $g_{kj}$ and find the optimal
$a_{ik}$ by weighted least squares, and then hold the $a_{ik}$ fixed
and find the optimal $g_{kj}$ by weighted least squares.  Each
iteration step is guaranteed to reduce the total $\chi^2$ and
converges in practice in ten or so iterations.  We initialize the
fitting with the output of a PCA on the spectra after the each
spectrum has been projected into the subspace orthogonal to the mean
spectrum of the data set; the first guesses for the $\gvec_k$ are set
to the mean spectrum and the first $K-1$ principal components.  We
find that performance is not worse if we initialize with a randomly
chosen set of spectra.

Importantly in the training, the wavelengths $\lambda_j$ are
rest-frame wavelengths, and because different training-set spectra
were taken at different redshifts, not all spectra have coverage at
all rest-frame wavelengths.  (There are also gaps from bad sky
subtraction, cosmic rays, and other data issues.)  Missing data in
spectrum $i$ at wavelength $j$ is handled naturally by setting
$1/\sigma^2_{ij}=0$; when a data point has vanishing inverse
variance, it does not contribute to $\chi^2$ or the fitting.  In
general, weighted least squares---because it uses the error variances
correctly---is unperturbed by missing data.

For any $K$-dimensional linear subspace, there are many choices for
the components $\gvec_k$: The components can be reordered, multiplied
by scalars, or replaced with linear combinations of themselves.  We
don't try to break all of these degeneracies, but we do enforce the
constraint that the $\gvec_k\cdot\gvec_k=1$, where the inner product
is taken with the trivial (identity) metric in the spectrum space.

\subsection{Test time}

At test time, we re-fit each \SDSS\ spectrum $i$ but now with two sets
of $K$ coefficients, one at the \SDSS\ redshift and one shifted in
velocity by a velocity difference $\Delta v$.  In order to simplify
redshift correction, we choose the (rest-frame) wavelength grid
$\lambda_j$ to be logarithmic
\begin{eqnarray}\displaystyle
\ln\lambda_j &=& \ln\lambda_0 + j\,\delta\ln\lambda
\quad ,
\end{eqnarray}
where $\ln\lambda_0$ is a constant and the logarithmic wavelength
interval is set to $\delta\ln\lambda=2.3026\times 10^{-4}$.  This
value is small enough that the spectroscopic wavelength element is
well sampled.  This permits trivial redshift analysis on a fine grid,
with redshift spacing $\delta\ln(1+z)=\delta z/[1+z]=\delta\ln\lambda$
or velocity spacing $\delta v=c\,\delta\ln\lambda=69.03~\kmps$.  As we
will see, finer velocity determinations can be obtained by quadratic
interpolation.

For any test spectrum $\lambda_i$, at any setting of second velocity
$v= n\,\delta v$, where $n$ is an integer, any vector $\avec$ of $K$
coefficients $a_k$ for the quasar spectrum at the \SDSS\ redshift, and
any vector $\bvec$ of $K$ coefficients $b_k$ for the second quasar
spectrum shifted relative to the \SDSS\ redshift by velocity
$n\,\delta v$, there is a goodness-of-fit scalar
$\chi^2_i(n,\avec,\bvec)$
\begin{eqnarray}\displaystyle
\chi^2_i(n,\avec,\bvec) &=& \sum_{j=1}^{M} \frac{1}{\sigma^2_{ij}}
 \,\left[f_{ij} - \sum_{k=1}^K a_k\,g_{kj} - \sum_{k=1}^K b_k\,g_{k[j-n]}\right]^2
\quad,
\end{eqnarray}
where implicitly the $1/\sigma^2_{ij}$ are set to vanish at any
wavelengths for which the relatively shifted spectral models don't
overlap.

\subsection{Visual inspection}

In principle, the question of whether a quasar spectrum is
significantly better fit by a pair of quasars than it is by a single
quasar is a straightforward condition on $\chi^2$, since $\chi^2$ is
simply related to the logarithm of the likelihood.  However, the
complexity of broad lines, artifacts in the data, intervening
absorbers, and strange quasars not well represented by the data-driven
model all lead to peaks in $\chi^2$ as a function of velocity
separation $n\,\delta v$ that do not in fact represent good fits to
the data.  For this reason, we applied simple criteria to the function
$\chi^2(n,\avec,\bvec)$ and inspected all candidates.

TBD:  What criteria did we use?  How many candidates did it produce?

TBD:  Show some examples of different categories.

\section{Data and results}

TBD:  How did we select SDSS spectra?

TBD:  Vivi make figures.

TBD:  Present catalog of binaries.

TBD:  Present catalog of double-peaked lines.

\section{Discussion}

TBD

\acknowledgements Thanks to people.  Thanks to \SDSS.  Thanks to
\project{R}.  Thanks to grants.

\end{document}
